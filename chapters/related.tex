\chapter{Related Work}\label{chapter:related}

Nowadays data mining landscape can be separated in stand-alone machine learning frameworks such as WEKA and ELKI, general-purpose scientific languages such as R, Julia or the Python library SciPy, and Big Data platforms such as Apache Hadoop and Spark.

\section{Research Tools}
Weka: Weka is a data mining software providing a large collection of machine learning algorithms for all aspects of data mining like data preprocessing, classification, regression, clustering, association rule mining and visualization. It is  actively developed by the University of Waikato and gained big attention in the machine learning community over the past decade. Weka is written in Java and can be used as a stand-alone GUI tool or executed directly within Java programs. 



ELKI: ELKI (Environment for Developing KDD-Applications Supported by Index Structures) is a relatively new data mining framework developed by the Ludwigs-Maximilians-Universität München to implement and evaluate algorithms in the field of data mining. ELKI provides the most important algorithms for data mining, is written in Java, and provides a GUI to be easily used by data scientists.  ELKI sees itself as an implementation framework for new data mining algorithms, leading to a better comparability among them and therefore to a fairer evaluation of the newly proposed algorithm. ELKI also encourages the use of index-structures to achieve performance gains when working with high-dimensional data sets.

\section{Tools for Statistical Computing}

: R is a language for statistical computing and data analysis, providing a variety of data mining libraries for all aspects of data mining and machine learning. It comes with rich graphical techniques and is therefore often researchers number one tool to create graphics for publications. Most of R libraries are written in R itself, however, C, C++ and Fortran code can be called at run time and is often used for computationally-intensive tasks. Unlike ELKI and Weka, R does not provide a graphical interface, however, there exist several projects, e.g. JGR or R Commander that aim to provide a R GUI.


SciPy: SciPy is a python environment providing several libraries to perform data mining such as NumPy, pandas and Matplotlib. As R, SciPy comes with algorithms for aggregation, clustering, classification and regression, all embedded in the Python language. Due to the elegance of the Python syntax, its popularity is growing, not only among data scientists, but also for prototyping new algorithms.

Julia: Julia is a relatively new dynamic programming language for scientific computing with a main purpose on high performance. As R, Julia is a programming language itself written mainly in Julia, as well as C and Fortran to gain better performance. For data analysis, external packages are available allowing the execution of state-of-the-art data mining algorithms. Interestingly, Julia uses LLVM-based just-in-time (JIT) compilation, and therefore is often matched with the performance of C. The same compilation technique is used in HyPer, therefore it will be interesting to compare both techniques.


\section{Big Data Platforms}

Apache Hadoop: Apache Hadoop is an open-source software for reliable, scalable, distributed computing of large datasets across clusters of computers. The heart of Hadoop is the Hadoop Distributed File System (HDFS) and Hadoop MapReduce, a simple programming model for distributed processing. Since MapReduce programs can be run on up to thousands of machines, it is ideal for Big Data. Several Algorithms can be performed on the MapReduce programming model, e.g. k-Means. For our research, it will be interesting to see how HyPer works with Big Data compared to Apache Hadoop.

Apache Spark: Apache Spark is a data analytics cluster computing framework, working on top of the Hadoop Distributed File System (HDFS). In contrast to Hadoop’s MapReduce, Spark comes with a richer programming model, leading to tremendous performance gains for some applications. Spark also provides in-memory cluster computing, making it well-suited to data mining algorithms. Regarding the main-memory capabilities of Spark, it will be interesting to compare Spark with HyPer.

\section{Databases KDD}
MDlib, rapid miner, opencv
