\chapter{Research Approach}\label{chapter:research}

While all of the presented environments are frequently used by data scientists, they demonstrate one decisive drawback: Before executing data mining algorithms on the datasets, the data first has to be fetched from the database and transformed into a format readable by these tools. With HyPer, we take a different approach: Instead of pulling the data from the database, we push the algorithms to the database. This leads to several advantages: A database system provides already efficient data storage and access, therefore data mining algorithms implemented on the database can benefit from these data structures. Besides, databases are optimized for modern hardware, e.g. multicore processors and cache hierarchies, which makes them presumably faster than platform-independent tools. Furthermore, data mining algorithms can profit from database features such as parallelization, scalability, recovery and backup facilities as well as the query language SQL. SQL itself comes already with useful algorithms for data analysis such as selection, sorting and aggregation. Therefore an extension of the query language to integrate other algorithms for data mining would feel very natural to the data scientist. 
Regarding those advantages, our research goal is to extend HyPer with data mining functionalities, exploiting the performance of a database for computational operations and building a general-purpose system for OLTP, OLAP and data mining queries. Such a system should outnumber above systems in both performance and usability.
As proof of concept, the well-known k-Means algorithm is used and will be implemented in HyPer. K-Means is one of the most popular data mining algorithms and available on all presented platforms. Since the algorithm is relatively simple and straightforward, a good comparability between different tools is given. 
Apart from performance evaluation, the implementation of k-Means should also demonstrate the possibilities of implementing data mining algorithms in HyPer and help to detect patterns and building blocks for further algorithms.



